\documentclass[journal,12pt,twocolumn]{IEEEtran}
\usepackage{setspace}
\usepackage{gensymb}
\singlespacing
\usepackage[cmex10]{amsmath}
\usepackage{amsthm}
\usepackage{mathrsfs}
\usepackage{txfonts}
\usepackage{stfloats}
\usepackage{bm}
\usepackage{cite}
\usepackage{cases}
\usepackage{subfig}
\usepackage{longtable}
\usepackage{multirow}
\usepackage{enumitem}
\usepackage{mathtools}
\usepackage{tikz}
\usepackage{circuitikz}
\usepackage{verbatim}
\usepackage[breaklinks=true]{hyperref}
\usepackage{tkz-euclide} % loads  TikZ and tkz-base
\usepackage{listings}
\usepackage{color}    
\usepackage{array}    
\usepackage{longtable}
\usepackage{calc}     
\usepackage{multirow} 
\usepackage{hhline}   
\usepackage{ifthen}   
\usepackage{lscape}     
\usepackage{chngcntr}
\DeclareMathOperator*{\Res}{Res}
\renewcommand\thesection{\arabic{section}}
\renewcommand\thesubsection{\thesection.\arabic{subsection}}
\renewcommand\thesubsubsection{\thesubsection.\arabic{subsubsection}}

\renewcommand\thesectiondis{\arabic{section}}
\renewcommand\thesubsectiondis{\thesectiondis.\arabic{subsection}}
\renewcommand\thesubsubsectiondis{\thesubsectiondis.\arabic{subsubsection}}
\renewcommand\thetable{\arabic{table}}
% correct bad hyphenation here
\hyphenation{op-tical net-works semi-conduc-tor}
\def\inputGnumericTable{}                                 %%

\lstset{
%language=C,
frame=single, 
breaklines=true,
columns=fullflexible
}
%\lstset{
%language=tex,
%frame=single, 
%breaklines=true
%}

\begin{document}
\newtheorem{theorem}{Theorem}[section]
\newtheorem{problem}{Problem}
\newtheorem{proposition}{Proposition}[section]
\newtheorem{lemma}{Lemma}[section]
\newtheorem{corollary}[theorem]{Corollary}
\newtheorem{example}{Example}[section]
\newtheorem{definition}[problem]{Definition}
\newcommand{\BEQA}{\begin{eqnarray}}
\newcommand{\EEQA}{\end{eqnarray}}
\newcommand{\define}{\stackrel{\triangle}{=}}
\bibliographystyle{IEEEtran}
\providecommand{\mbf}{\mathbf}
\providecommand{\pr}[1]{\ensuremath{\Pr\left(#1\right)}}
\providecommand{\qfunc}[1]{\ensuremath{Q\left(#1\right)}}
\providecommand{\sbrak}[1]{\ensuremath{{}\left[#1\right]}}
\providecommand{\lsbrak}[1]{\ensuremath{{}\left[#1\right.}}
\providecommand{\rsbrak}[1]{\ensuremath{{}\left.#1\right]}}
\providecommand{\brak}[1]{\ensuremath{\left(#1\right)}}
\providecommand{\lbrak}[1]{\ensuremath{\left(#1\right.}}
\providecommand{\rbrak}[1]{\ensuremath{\left.#1\right)}}
\providecommand{\cbrak}[1]{\ensuremath{\left\{#1\right\}}}
\providecommand{\lcbrak}[1]{\ensuremath{\left\{#1\right.}}
\providecommand{\rcbrak}[1]{\ensuremath{\left.#1\right\}}}
\theoremstyle{remark}
\newtheorem{rem}{Remark}
\newcommand{\sgn}{\mathop{\mathrm{sgn}}}
\providecommand{\abs}[1]{\left\vert#1\right\vert}
\providecommand{\res}[1]{\Res\displaylimits_{#1}} 
\providecommand{\norm}[1]{\left\lVert#1\right\rVert}
\providecommand{\mtx}[1]{\mathbf{#1}}
\providecommand{\mean}[1]{E\left[ #1 \right]}
\providecommand{\fourier}{\overset{\mathcal{F}}{ \rightleftharpoons}}
\providecommand{\system}[1]{\overset{\mathcal{#1}}{ \longleftrightarrow}}
\newcommand{\solution}{\noindent \textbf{Solution: }}
\newcommand{\cosec}{\,\text{cosec}\,}
\providecommand{\dec}[2]{\ensuremath{\overset{#1}{\underset{#2}{\gtrless}}}}
\newcommand{\myvec}[1]{\ensuremath{\begin{pmatrix}#1\end{pmatrix}}}
\newcommand{\mydet}[1]{\ensuremath{\begin{vmatrix}#1\end{vmatrix}}}
\renewcommand{\vec}[1]{\boldsymbol{\mathbf{#1}}}
\def\putbox#1#2#3{\makebox[0in][l]{\makebox[#1][l]{}\raisebox{\baselineskip}[0in][0in]{\raisebox{#2}[0in][0in]{#3}}}}
     \def\rightbox#1{\makebox[0in][r]{#1}}
     \def\centbox#1{\makebox[0in]{#1}}
     \def\topbox#1{\raisebox{-\baselineskip}[0in][0in]{#1}}
     \def\midbox#1{\raisebox{-0.5\baselineskip}[0in][0in]{#1}}

\vspace{3cm}
\title{PT-100 }
\author{Shristy Sharma}
\maketitle
\tableofcontents
\bigskip

\begin{abstract}
    This document contains a lab report on the modeling of the voltage-temperature
    characteristics of the PT-100 RTD (Resistance Temperature Detector) using
    least squares method.
\end{abstract}

\section{Training Data}
Training data obtained by PT-100 to train the Arduino is shown in Table
\ref{tab:Training data }.

\begin{table}[!ht]
    \centering
    \input{tables/training_data.tex}
    \caption{Training data.}
    \label{tab:Training data }
\end{table}

The effective schematic circuit diagram is shown in 
\autoref{fig:circuit}.

\begin{figure}[!ht]
    \centering
    \begin{circuitikz} \draw
        (0,0) to[battery1, l=$3.3\ V$, invert] (0,2)
        to[R, l^=$32\ \Omega$] (3,2) to[short, -o] (5,2);
        \draw (3,2) to[R, l^=$P\ \Omega$] (3,0)
        -- (0,0);
        \draw (3,0) to[short, -o] (5,0);
    \end{circuitikz}
    \caption{Schematic Circuit Diagram to Measure the Output of PT-100 ($P$).}
    \label{fig:circuit}
\end{figure}

\section{Model}

For the PT-100, we use the Callendar-Van Dusen equation
\begin{align}
    V(T) &= V(0)\brak{1+AT+BT^2} \\
    \implies c &= \vec{n}^\top\vec{x} \label{eq:model}
\end{align}
where
\begin{align}
    c = V(T),\ \vec{n} = V(0)\myvec{1\\A\\B},\ \vec{x} = \myvec{1\\T\\T^2}
    \label{eq:x-y-theta-def}
\end{align}

For multiple points, \eqref{eq:model} becomes
\begin{align}
    \vec{X}^\top\vec{n} = \vec{C}
    \label{eq:lsq-eqn}
\end{align}
where
\begin{align}
    \vec{X} &= \myvec{1&1&\ldots&1\\T_1&T_2&\ldots&T_n\\T_1^2&T_2^2&\ldots&T_n^2} \\
    \vec{C} &= \myvec{V\brak{T_1}\\V\brak{T_2}\\\vdots\\V\brak{T_n}}
\end{align}
and $\vec{n}$ is the unknown.

\section{Solution}
We approximate $\vec{n}$ by using the least sqaures method. The Python code 
\texttt{codes/pt100.py} solves for $\vec{n}$.

The calculated value of $\vec{n}$ is
\begin{align}
    \vec{n} = \myvec{2.5577569\\2.0663864\times10^{-3}\\-2.9546268\times10^{-6}}
    \label{eq:opt-theta}
\end{align}

The approximation is shown in Fig. \ref{fig:Training data}.
\begin{figure}[!ht]
    \centering
    \includegraphics[width=\columnwidth]{figs/train.png}
    \caption{Training the model.}
    \label{fig:Training data}
\end{figure}

Thus, the approximate model is given by
\begin{align}
    V(T) &= 2.5577569 + \brak{2.0663864\times10^{-3}}T \nonumber \\
         &- \brak{2.9546268\times10^{-6}}T^2 
    \label{eq:opt-model}
\end{align}
Equation 8 can be written in the form of,
\begin{align}
ax^2 + bx + c=0\\
\implies 2.9546268\times10^{-6}T^2 + 2.0663864\times10^{-3}T \nonumber \\
- \brak{2.5577569 - V(T)} = 0 
\end{align}  
Now, we can use the quadratic formula to find the value of the temperature.(which has been done in Arduino)
\section{Validation}
The validation dataset is shown in Table \ref{tab:validation data}. The results of the 
validation are shown in Fig. \ref{fig:validation fig}.
\begin{table}[!ht]
    \centering
    \input{tables/validation_data.tex}
    \caption{Validation data.}
    \label{tab:validation data}
\end{table}
\begin{figure}[!ht]
    \centering
    \includegraphics[width=\columnwidth]{figs/valid.png}
    \caption{Validating the model.}
    \label{fig:validation fig}
\end{figure}
\section{Conclusion}
The interfacing has been done between Arduino and LCD where, the temperature has been displayed.
The modelling of the sensor has been done using Python and has been executed using a microcontroller.
This project demonstrates how machine learning methods can be used to 
model the behaviour of an unknown component, and find the right parameters that 
fit the model. 
\end{document}
