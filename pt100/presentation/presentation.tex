%%%%%%%%%%%%%%%%%%%%%%%%%%%%%%%%%%%%%%%%%%%%%%%%%%%%%%%%%%%%%%%
%
% Welcome to Overleaf --- just edit your LaTeX on the left,
% and we'll compile it for you on the right. If you open the
% 'Share' menu, you can invite other users to edit at the same
% time. See www.overleaf.com/learn for more info. Enjoy!
%
%%%%%%%%%%%%%%%%%%%%%%%%%%%%%%%%%%%%%%%%%%%%%%%%%%%%%%%%%%%%%%%


% Inbuilt themes in beamer
\documentclass{beamer}
\usepackage{setspace}
\usepackage{gensymb}
\singlespacing
\usepackage{amsmath}
\usepackage{bm}
\usepackage{cite}
\usepackage{cases}
\usepackage{subfig}
\usepackage{longtable}
\usepackage{multirow}
\usepackage{verbatim}
\usepackage{hyperref}
\usepackage{listings}
\usepackage{color}    
\usepackage{array}    
\usepackage{longtable}
\usepackage{calc}     
\usepackage{multirow} 
\usepackage{hhline}   
\usepackage{ifthen}   
\DeclareMathOperator*{\Res}{Res}

% correct bad hyphenation here
\hyphenation{op-tical net-works semi-conduc-tor}
\def\inputGnumericTable{}                                 %%

\lstset{
%language=C,
frame=single, 
breaklines=true,
columns=fullflexible
}
%\lstset{
%language=tex,
%frame=single, 
%breaklines=true
%}
\hypersetup{
    colorlinks=true,
    linkcolor=black,
    urlcolor=blue,
}
\usetheme{CambridgeUS}

\DeclareMathOperator*{\argmax}{arg\,max}
\DeclareMathOperator*{\argmin}{arg\,min}
\newtheorem{proposition}{Proposition}[section]
\newcommand{\BEQA}{\begin{eqnarray}}
\newcommand{\EEQA}{\end{eqnarray}}
\newcommand{\define}{\stackrel{\triangle}{=}}
\bibliographystyle{IEEEtran}
\providecommand{\mbf}{\mathbf}
\providecommand{\pr}[1]{\ensuremath{\Pr\left(#1\right)}}
\providecommand{\qfunc}[1]{\ensuremath{Q\left(#1\right)}}
\providecommand{\sbrak}[1]{\ensuremath{{}\left[#1\right]}}
\providecommand{\lsbrak}[1]{\ensuremath{{}\left[#1\right.}}
\providecommand{\rsbrak}[1]{\ensuremath{{}\left.#1\right]}}
\providecommand{\brak}[1]{\ensuremath{\left(#1\right)}}
\providecommand{\lbrak}[1]{\ensuremath{\left(#1\right.}}
\providecommand{\rbrak}[1]{\ensuremath{\left.#1\right)}}
\providecommand{\cbrak}[1]{\ensuremath{\left\{#1\right\}}}
\providecommand{\lcbrak}[1]{\ensuremath{\left\{#1\right.}}
\providecommand{\rcbrak}[1]{\ensuremath{\left.#1\right\}}}
\theoremstyle{remark}
\newtheorem{rem}{Remark}
\newcommand{\sgn}{\mathop{\mathrm{sgn}}}
\providecommand{\abs}[1]{\left\vert#1\right\vert}
\providecommand{\res}[1]{\Res\displaylimits_{#1}} 
\providecommand{\norm}[1]{\left\lVert#1\right\rVert}
\providecommand{\mtx}[1]{\mathbf{#1}}
\providecommand{\mean}[1]{\mathbb{E}\left[ #1 \right]}   
\providecommand{\fourier}{\overset{\mathcal{F}}{ \rightleftharpoons}}
\providecommand{\system}[1]{\overset{\mathcal{#1}}{ \longleftrightarrow}}
\newcommand{\cosec}{\,\text{cosec}\,}
\providecommand{\dec}[2]{\ensuremath{\overset{#1}{\underset{#2}{\gtrless}}}}
\newcommand{\myvec}[1]{\ensuremath{\begin{pmatrix}#1\end{pmatrix}}}
\newcommand{\mydet}[1]{\ensuremath{\begin{vmatrix}#1\end{vmatrix}}}
\renewcommand{\vec}[1]{\mathbf{\boldsymbol{#1}}}
% Theme choice:
\newcounter{saveenumi}
\newcommand{\seti}{\setcounter{saveenumi}{\value{enumi}}}
\newcommand{\conti}{\setcounter{enumi}{\value{saveenumi}}}

\resetcounteronoverlays{saveenumi}
% Title page details: 
\title[PT-100 ]{An Application of Machine Learning to model a Temperature Sensor(PT100)} 
\author{Shristy Sharma}
\begin{document}

% Title page frame
\begin{frame}
    \titlepage 
\end{frame}

% Remove logo from the next slides
\logo{}

% Outline frame
\begin{frame}{Outline}
    \tableofcontents
\end{frame}


% Lists frame
\section{Introduction}
\begin{frame}{Aim}
The modeling of the voltage-temperature characteristics of the PT-100 RTD (Resistance Temperature Detector) using least squares method.

\end{frame}
\section{Circuit Diagram}
 \begin{frame}{Circuit Diagram}
 
 \begin{figure}[!ht]
        \centering
        \includegraphics[width=0.6\columnwidth]{figs/ckt.png}
    \end{figure}
\end{frame}

\section{Data}
 \begin{frame}{Training data}
\begin{table}[!ht]
    \centering
    \input{tables/training_data.tex}
    \caption{Training data}
    \label{tab:Training data }
\end{table}
\end{frame}
 \begin{frame}{Validation data}
\begin{table}[!ht]
    \centering
    \input{tables/validation_data.tex}
    \caption{Validation data}
    \label{tab:Validation data}
\end{table}
\end{frame}

\section{Model}
\begin{frame}{Model}
For the PT-100, we use the Callendar-Van Dusen equation
\begin{align}
    V(T) &= V(0)\brak{1+AT+BT^2} \\
    \implies c &= \vec{n}^\top\vec{x}\\
    c = V(T),\ \vec{n} = V(0)\myvec{1\\A\\B},\ \vec{x} = \myvec{1\\T\\T^2}
\end{align}
\end{frame}
\begin{frame}{Model}
For multiple points,eqn (3) becomes
   \begin{align}
    \vec{X}^\top\vec{n} = \vec{C}\\
    \vec{X} &= \myvec{1&1&\ldots&1\\T_1&T_2&\ldots&T_n\\T_1^2&T_2^2&\ldots&T_n^2} \\
    \vec{C} &= \myvec{V\brak{T_1}\\V\brak{T_2}\\\vdots\\V\brak{T_n}}
      \end{align}
and $\vec{n}$ is the unknown.
\end{frame}
\begin{frame}{Model}
We approximate $\vec{n}$ by using the least sqaures method. The Python code 
\texttt{codes/pt100.py} solves for $\vec{n}$.
The calculated value of $\vec{n}$ is
\begin{align}
    \vec{n} = \myvec{2.5577569\\2.0663864\times10^{-3}\\-2.9546268\times10^{-6}}
\end{align}
The approximation is shown in Figures further.
\end{frame}
\begin{frame}{Model}
Thus, the approximate model is given by
\begin{align}
    V(T) &= 2.5577569 + \brak{2.0663864\times10^{-3}}T \nonumber \\
         &- \brak{2.9546268\times10^{-6}}T^2 
\end{align}

Equation 8 can be written in the form of,
\begin{align}
ax^2 + bx + c=0\\
\implies 2.9546268\times10^{-6}T^2 + 2.0663864\times10^{-3}T \nonumber \\
- \brak{2.5577569 - V(T)} = 0 
\end{align}  
Now, we can use the quadratic formula to find the value of the temperature.(which has been done in Arduino)
\end{frame}


\section{Data Visualisation}
 \begin{frame}{Data Visualization}
    \begin{figure}[!ht]
        \centering
        \includegraphics[width=0.6\columnwidth]{figs/train.png}
        \caption{TRAINING DATA}
        \label{fig:data}
    \end{figure}
\end{frame}

 \begin{frame}{Data Visualization}
    \begin{figure}[!ht]
        \centering
        \includegraphics[width=0.6\columnwidth]{figs/valid.png}
        \caption{VALIDATION DATA .}
        \label{fig:data}
    \end{figure}
\end{frame}
\begin{frame}{Experiment}
   \begin{figure}[!ht]
        \centering
        \includegraphics[width=0.6\columnwidth]{figs/arduino.png}
    \end{figure}
\end{frame}
\section{Conclusions}
\begin{frame}{Conclusions}
\begin{enumerate}
\item The modelling of the sensor has been done using Python and has been executed using a microcontroller.
\item This project demonstrates how machine learning methods can be used to model the behaviour of an unknown component, and find the right parameters that fit the model.
\end{enumerate}
\end{frame}
\end{document}
